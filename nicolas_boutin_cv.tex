% Preamble

\documentclass[11pt, a4paper, sans]{moderncv}

\moderncvstyle{casual}
\moderncvcolor{black}
\setlength{\hintscolumnwidth}{2cm} % date column width

\usepackage[utf8]{inputenc}% encodage, à modifier selon vos habitudes
\usepackage[frenchb]{babel}% pour un document en français.
\usepackage[scale=0.8]{geometry}% pour régler les marges du CV les options habituelles de l'extension geometry peuvent s'appliquer ici

\name{Nicolas}{Boutin}
\title{26/04/1986, 32ans}
\address{4 rue du chausseur}{86110 Amberre}
\phone[mobile]{06~79~37~39~00}
\email{boutwork@gmail.com}
%\homepage{www.pierredurand.com}
%\social[linkedin]{pierre.durand}
%\social[twitter]{pierre.durand}
%\social[github]{pierre.durand}
%\extrainfo{informations complémentaires.}
%\quote{Encore un titre}
%\photo[64pt][0.4pt]{maphoto} % hauteur, épaisseur cadre, pathname

% Document

\begin{document}

\makecvtitle

\section{Compétences}

\cvdoubleitem{Architecture, Conception et Développement logiciel}{}{Test / Intégration}{Jenkins, Catch2, Google Test}
\cvdoubleitem{Programmation/Modélisation}{C99, C++03/11/14/17, CMake, Python, Perl, Bash, UML, Design pattern}{Management}{Pilotage d'équipe, Cycle en V}
\cvdoubleitem{Système embarqué}{STM32, ARM M3-M4, FreeRTOS, RL-RTX, Bootloader, UART, SPI, I2C, RTC, PWM, Segger emWin, emFile, emUSB}
	{Logiciel/IDE}{Git, Perforce, SVN, Eclipse, IAR, Keil uVision, STM32CubeMX, Bureautique, JIRA}
\cvdoubleitem{Linux embarqué}{Yocto, bitbake, SystemV, IPC, DBus}{Anglais}{TOEIC 895}

\section{Expériences professionnelles}

\section{Formation}

%cventry % on peut mettre ici de 3 à 6 arguments qui peuvent être laissés vides
\cventry{2008--2011}{Diplôme Ingénieur Sciences Informatiques}{options Systèmes Embarqués}{Polytech'Nice-Sophia}{Sophia-Antipolis(06)}{}
\cventry{2007--2008}{Diplôme BSc(Hons) Robotics (Bac+4)}{Université de Plymouth}{Angleterre}{}{}{}
\cventry{2005--2007}{Diplôme Universitaire de Technologie}{DUT Génie Electrique et Informatique Industrielle}{Angers(49)}{}{}
\cventry{2004--2005}{Baccalauréat Sciences et Techniques Industrielles}{STI Génie Electronique}{St-Julien-la-Baronnerie}{Angers(49)}{}

\section{Centres d'intérêts}

\cvitem{Sports}{Trail/Course nature, Voile, Badminton}
\cvitem{Lecture}{Actualités technologique, scientifiques, robotiques et spatiales}
\cvitem{Robotique}{Coupe de France de robotique 2007, Ferté-Bernard. Commande moteurs (Asservissement position et vitesse)}

\end{document}
