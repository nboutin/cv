% Preamble

\documentclass[10pt, a4paper, sans]{moderncv}

\moderncvtheme[blue]{classic} % blue, orange, green, red, purple, grey, black / classic, casual, oldstyle, banking
%\moderncvstyle{casual}
%\moderncvcolor{black}
\setlength{\hintscolumnwidth}{2cm} % date column width

\usepackage[utf8]{inputenc}% encodage, à modifier selon vos habitudes
\usepackage[frenchb]{babel}% pour un document en français.
%\usepackage[scale=0.8]{geometry}% pour régler les marges du CV les options habituelles de l'extension geometry peuvent s'appliquer ici
\usepackage[top=1.1cm, bottom=1.1cm, left=1cm, right=1cm]{geometry}
\recomputelengths
\nopagenumbers{} %suppress page numbering for CV longer than one page

% Personal data

\name{Nicolas}{Boutin}
\title{Ingénieur logiciel embarqué\newline \large 8 années d'expériences}
\address{4 rue du chausseur}{86110 Amberre}
\phone[mobile]{06~79~37~39~00}
\email{boutwork@gmail.com}
%\homepage{www.pierredurand.com}
%\social[linkedin]{pierre.durand}
%\social[twitter]{pierre.durand}
%\social[github]{pierre.durand}
\extrainfo{26/04/1986, 32 ans}
%\quote{Ingénieur logiciel embarqué}
%\photo[64pt][0.4pt]{maphoto} % hauteur, épaisseur cadre, pathname (sans extension)

% Document

\begin{document}

\makecvtitle
% Marge négative entre le titre et la partie expérience, pour gagner de la place
%\vspace*{-2.5\baselineskip}

\section{Compétences}

\cvcomputer{Logiciel}{Architecture, Conception et Développement}
	{Management}{Pilotage d'équipe, Cycle en V}
\cvcomputer{Programmation}{C99, C++03/11/14/17, CMake, Doxygen, GNU Debugger(GDB), Python, Perl, Bash}
	{Modélisation}{UML, Design pattern}
\cvcomputer{Système embarqué}{STM32, ARM M3-M4, FreeRTOS, RL-RTX, Bootloader, UART, SPI, I2C, RTC, PWM, Segger emWin, emFile, emUSB}
	{Linux embarqué}{Yocto, bitbake, SystemV, IPC, DBus, CommonAPI}
\cvcomputer{Logiciel/IDE}{Git, Perforce, SVN, Eclipse, IAR, Keil uVision, STM32CubeMX, Bureautique, JIRA}
	{Test\newline Intégration}{Jenkins, Catch2, Google Test, Parasoft, Cppcheck, Travis CI}
\cvcomputer{Anglais}{TOEIC 895}{}{}

\section{Expériences professionnelles}

%Mai. 2017 Ingénieur logiciel linux embarqué, Groupe Creative Rennes pour Magneti Marelli, Châtellerault (86).
%Maintenant Développement et portage applications C++ de monitoring et de services OS.
%Yocto, Bitbake, DLT logs, CoreDump Handler, IPC DBUS, CMake
\cventry{2017--2019\newline 2 ans}{Ingénieur logiciel linux embarqué}{Groupe Creative Rennes pour Magneti Marelli}{Châtellerault(86)}
	{Maintenant Développement et portage applications C++ de monitoring et de services OS}
	{Yocto, Bitbake, DLT logs, IPC DBUS, GDB, CMake, Eclipse}
	
%Oct. 2016 Ingénieur intégrateur, mise en place de l’intégration continue, Apside Rennes pour Faiveley, Tours (37).
%Janv. 2017 Ubuntu14.04, Jenkins, SVN, Script Shell, Makefile, Perl
\cventry{2016--2017\newline 4 mois}{Ingénieur intégrateur}{Apside Rennes pour Faiveley, Tours(37)}{Mise en place de l’intégration continue}
	{Ubuntu14.04, Jenkins, SVN, Script Shell, Makefile, Perl}{}

%Oct. 2015 Architecte logiciel embarqué temps réel, Apside Rennes pour Canberra, Loches (37).
%Juil. 2016 Définition et implémentation de l’architecture logiciel pour contaminamètre de chantier. Pilotage développement logiciel équipe 3 ingénieurs.
%C99/C++03, IAR, Eclipse, SVN, STM32CubeMX, FreeRTOS, STM32L4, Segger emWin, emFile, emUSB
\cventry{2015--2016\newline 10 mois}{Architecte logiciel embarqué temps réel}{Apside Rennes pour Canberra, Loches(37)}
	{Définition et implémentation de l’architecture logiciel pour contaminamètre de chantier}{Pilotage développement logiciel équipe 3 ingénieurs}
	{C99/C++03, IAR, Eclipse, SVN, STM32CubeMX, FreeRTOS, STM32L4, Segger emWin, emFile, emUSB}

%Mai. 2014 Responsable logiciel embarqué du projet Linky G1 Triphasé, Itron, Poitiers (86).
%Sept. 2015 Pilotage développement logiciel et management d’équipe, 3 ingénieurs. Report d’activité et avancement au chef de projet. Conception et développement. Mise en place de test d’intégration. Support technique.
\cventry{2015--2015\newline 9 mois}{Responsable logiciel embarqué du projet Linky G1 Triphasé}{Itron}{Poitiers(86)}
	{Pilotage développement logiciel et management d’équipe, 3 ingénieurs. Report d’activité et avancement au chef de projet}
	{Conception et développement. Mise en place de test d’intégration. Support technique}

%Oct. 2011 Ingénieur logiciel embarqué temps réel sur compteur électrique, Itron, Poitiers (86).
%Juin 2014 Conception et développement de fonctionnalités spécifiques du domaine.
%C++03, STM32, RTOS RL-RTX, Eclipse, Keil uVision, Perforce
\cventry{2011--2014\newline 3 ans}{Ingénieur logiciel embarqué temps réel}{Itron}{Poitiers(86)}{Conception et développement de fonctionnalités pour compteur électrique Linky}
	{C++03, STM32, RTOS RL-RTX, Eclipse, Keil uVision, Perforce}

%Sept. 2010 Apprenti par alternance, ST-Ericsson, Sophia-Antipolis (06).
%Sept. 2011 Conception et développement d’un système de diagnostique matériel pour ARM9 embarqué dans les téléphones.
%C/C++, Trace32, Lauterbach, Eclipse, QT 4
\cventry{2010--2011}{Apprenti par alternance}{ST-Ericsson}{Sophia-Antipolis (06)}
	{Conception et développement d’un système de diagnostique matériel pour ARM9 embarqué dans les téléphones}
	{C99/C++03, Trace32, Lauterbach, Eclipse, QT 4}

\section{Formations}

%cventry % on peut mettre ici de 3 à 6 arguments qui peuvent être laissés vides
%\cventry{Avril 2012\\à Octobre 2012}{Administrateur systèmes}{Vesalis}{Clermont-Ferrand}{France}{}
\cventry{2008--2011}{Diplôme Ingénieur Sciences Informatiques}{options Systèmes Embarqués}{Polytech'Nice-Sophia}{Sophia-Antipolis(06)}{}
\cventry{2007--2008}{Diplôme BSc(Hons) Robotics (Bac+4)}{Université de Plymouth}{Angleterre}{}{}{}
\cventry{2005--2007}{Diplôme Universitaire de Technologie}{DUT Génie Electrique et Informatique Industrielle}{Angers(49)}{}{}
\cventry{2004--2005}{Baccalauréat Sciences et Techniques Industrielles}{STI Génie Electronique}{St-Julien-la-Baronnerie}{Angers(49)}{}

\section{Centres d'intérêts}

\cvitem{Sports}{Trail/Course nature, Voile, Badminton}
\cvitem{Lecture}{Actualités technologique, scientifiques, robotiques et spatiales}
\cvitem{Robotique}{Coupe de France de robotique 2007, Ferté-Bernard. Commande moteurs (Asservissement position et vitesse)}

% Footnote at the end to add version
\emptysection{}\closesection
\vfill
\begin{center}
\textit{\small 2019.02-rc}
\end{center}

\end{document}
