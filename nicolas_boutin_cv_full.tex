% Preamble

\documentclass[10pt, a4paper, sans]{moderncv}

\moderncvtheme[blue]{classic} % blue, orange, green, red, purple, grey, black / classic, casual, oldstyle, banking
%\moderncvstyle{casual}
%\moderncvcolor{black}

% quotewidth, separatorcolumnwidth, maincolumnwidth, doublei-temmaincolumnwidth, listitemsymbolwidth, listdoubleitemmaincolumnwidth,
\setlength{\hintscolumnwidth}{2.1cm} % date column width

\usepackage[utf8]{inputenc}% encodage, à modifier selon vos habitudes
\usepackage[frenchb]{babel}% pour un document en français.
%\usepackage[scale=0.8]{geometry}% pour régler les marges du CV les options habituelles de l'extension geometry peuvent s'appliquer ici
\usepackage[top=1.1cm, bottom=1.1cm, left=1cm, right=1cm]{geometry}
\recomputelengths
\nopagenumbers{} %suppress page numbering for CV longer than one page

% Personal data

\name{Nicolas}{Boutin}
\title{Ingénieur logiciel embarqué\newline \large 10 années d'expériences}
\address{4 Rue du Chausseur}{86110 Amberre}
\phone[mobile]{06~79~37~39~00}
\email{boutwork@gmail.com}
%\homepage{www.pierredurand.com}
%\social[linkedin]{pierre.durand}
%\social[twitter]{pierre.durand}
%\social[github]{pierre.durand}
\extrainfo{26/04/1986, 35 ans}
%\quote{Ingénieur logiciel embarqué}
%\photo[64pt][0.4pt]{maphoto} % hauteur, épaisseur cadre, pathname (sans extension)

% Document

\begin{document}

\makecvtitle
% Marge négative entre le titre et la partie expérience, pour gagner de la place
%\vspace*{-2.5\baselineskip}

\section{Expériences professionnelles}

%\cventry{years}{degree/job title}{institution/employer}{localization}{optional: grade/...}{optional: comment/job description}
%cventry % on peut mettre ici de 3 à 6 arguments qui peuvent être laissés vides

% Fevrier 2019
% PMAM (LLD, dev, UT, IT)(DaVinci configurator)
% Flashing (dev, test, bootloader, agent, ota_maker)
% Bring up (MP)
% IMIcro communication(maintenance)
% Setup UT env, support tech to India
% MAF (outil débarqué)
\cventry{2019--2021\newline2 ans}{Ingénieur logiciel micro-controlleur}{Groupe Creative Rennes pour Magneti Marelli}{Châtellerault(86)}
	{Architecture, Développement et Tests composant logiciel embarqué en C. Architecture et Développement outils débarqués pour banc de test en Python}
	{C99, Python3, Renesas MCAL (Microcontroller Abtraction Layer), Vector DaVinci configurator, MicroSAR OS}

% AIDA Flashing
\subsection{Contexte}
\cvline{}{
\begin{itemize}
  \item Projet AIDA, produit infocluster, tableau de bord et info-multmédia dans la voiture (Stellantis, P308, DS3)
  \item Equipe OSS, 1 Domain leader, 2 Architect, 4 developpeurs
  \item Vuc (Vehicule Controler) Microcontroleur RH850, Renesas MCAL Vector MicroSAR OS
  \item SoC (System on Chip) Qualcomm SA8150 (8coeur), Hypervisor GHS (Green Hills), Guest Linux ou Android
\end{itemize}}

\subsection{Activités}
\cvline{}{
\begin{itemize}
  \item Flashing du Vuc depuis le SOC (communication intermicro SPI)
  \item 3 partitions à mettre à jour (Bootloader secure, Agent, Application) et 3 scénario (Update, upgrade, migration)
  \item Developpement Secureboot, selection de partition, controle d'intégrité des partitions
  \item Developpement partition Agent/Updater, mise à jour de la partition application ou agent(updater)
  \item Developpement composant Enabler (Application), démarrage de la séquence de flashing, request reboot to agent
  \item Type de package Header(partition dest, size, signé et/ou crypté)
  \item outils débarqué pour signer, crypter, fill blank, convert from hex to bin, ...
\end{itemize}}

% Mai 2017
\cventry{2017--2019\newline2 ans}{Ingénieur logiciel linux embarqué}{Groupe Creative Rennes pour Magneti Marelli}{Châtellerault(86)}
	{Développement et portage applications C++ de monitoring et de services OS}
	{C++14, Linux User-Space, Yocto, Bitbake, CMake, DLT logs, IPC DBUS, GDB, Eclipse}
	
% TBM2 2017/05-2019-03
\subsection{Contexte}
\cvline{}{
\begin{itemize}
  \item Projet TBM2, Antenne intelligente, fonctionnalité Appel d’urgence, d'assistance, Anti vol, Wifi dans le véhicule via 4G,\ldots
  \item Equipe Operating System Service (OS Service), 1 Domain Leader, 1 Architect, 6 Developper
  \item Distribution linux SWI Legato avec fonctionnalité automobile intégré
  \item Legato=open source Linux-based embedded platform designed to simplify connected IoT application
  \item SWI Air Prime AR758x (Baseband MDM9x40, Cortex A7, 1.2GHz, 145M RAM,512M Flash)
  \item Yocto 2.2.3, système de build linux, kernel 3.18, Bitbake (intégré à Yocto) séquenceur de tache pour le build
  \item Linux User Space, GNU Debugger (GDB), Unix System V
  \item DBUS, CommonAPI, DLT-Daemon (log), coredump handler, adb shell, fastboot (flashing)
\end{itemize}}

\subsection{Activités}
\cvline{}{
\begin{itemize}
  \item Scrum matinale quotidien en Anglais (Equipe Italienne)
  \item NRE-Imicro-Server: Server de communication inter-micro (SPI) et Dbus, Cheetah code génération, C++/Dbus/CommonAPI
  \item dlt-serial: Transfer data from UART to dlt-daemon
  \item db-manager: C++ wrapper over sqlite3 + contrainte domaine
  \item earlystarter: first process in user-space, get data from Vuc, mount partition, launch OSS application, start initd
\end{itemize}}

% AIDA Flashing 2019/04-2019/11
\subsection{Contexte}
\cvline{}{
\begin{itemize}
  \item Projet AIDA, Fusion Tableau de bord et Infotainement(Multimédia)
  \item Responsabilité OSS, Mettre à jour le Véhicule Controller via OTA
  \item Perimetre, Lib SoC et composant logiciel Vuc (bootloader, Agent, Application)
  \item 1 architect, 1 domain leader, 3 developper, 8 mois
  \item µC Renesas RH850, langage C99, AUTOSAR, Green Hills Software (GHS) toolchain
\end{itemize}}

\subsection{Activités}
\cvline{}{
\begin{itemize}
  \item Scrum matinale quotidien
  \item Rapport d'activité hedbomadaire au Fonction Leader(chef équipe)
  \item Développement et intégration de la fonction Flashing Vuc
  \item Développement d'un framework de test d'intégration basé sur python unittest(2 personnes, 4 semaines)
  \item Définition architecture, décomposition en tache, évaluation de la charge, attribution activité
\end{itemize}}

% AIDA PMAM 2019/12
\subsection{Context}
\cvline{}{
\begin{itemize}
  \item Projet AIDA R1, Fusion Tableau de bord et Infotainement(Multimédia)
  \item Contrainte Safety, AsilA (ISO26262:2011 for Road Vehicle)
  \item Règle MISRA Safety (Parasoft)
\end{itemize}}

\subsection{Activités}
\cvline{}{
\begin{itemize}
  \item PMAM Post-Mortem Analysis Module
  \item Ecriture document SwDD (Software Details Design)
  \item Definition architecture (Enterprise Architect), Diagramme de classe, séquence,
  \item Implémentation
  \item Test Unitaire (Google Test)
\end{itemize}}
	
% Oct 2016 - Janv 2017
\cventry{2016--2017\newline4 mois}{Ingénieur intégrateur}{Apside Rennes pour Faiveley}{Tours(37)}
	{Mise en place de l’intégration continue}
	{Ubuntu14.04, Jenkins, SVN, Script Shell, Makefile, Perl}
\subsection{Contexte}
\cvline{}{
\begin{itemize}
  \item Ordinateur de bord pour train: Tachymètre, Gestion des caméras intérieur, Enregistreur de parcours, Boite noire
\end{itemize}}
\subsection{Activités}
\cvline{}{
\begin{itemize}
  \item Configuration du serveur Jenkins/Hudson
  \item Compilation multi-projet: Plateforme HW x3 / Repo multi-projet « Générique » / Projet spécifique
  \item Lancement du séquenceur de test et du pilotage du banc de Test, Pilotage d’alimentation, Connexion réseau
\end{itemize}}

% Oct 2015 - Juil 2016
\cventry{2015--2016\newline10 mois}{Architecte logiciel embarqué temps réel}{Apside Rennes pour Canberra}{Loches(37)}
	{Définition et implémentation de l’architecture logiciel pour contaminamètre de chantier. Pilotage développement logiciel équipe 3 ingénieurs}
	{C99/C++03, FreeRTOS, STM32L4, STM32CubeMX, IAR, Eclipse, SVN, Segger emWin, emFile, emUSB}
	
\subsection{Contexte}
\cvline{}{
\begin{itemize}
  \item Contaminamètre de chantier: Appareil portatif, Avec écran LCD et 5 boutons, Mesure de radioactivité avec sonde interchangeable (alpha, beta, gamma)
\end{itemize}}
\subsection{Activités}
\cvline{}{
\begin{itemize}
  \item Rédaction document spécifications fonctionnelles
  \item Rédaction document d’architectures logiciels, modélisation UML
  \item Conception et développement de l’architecture logiciel du produit (Composant = modularité)
  \item Conception et développement Bootloader: Fonction firmware update, Lecture clé USB (maitre), Transfert PC sur flash interne (esclave), Mécanisme de récupération sur échec de mise à jour
  \item Implémentation du driver RTC
  \item Implémentation wrapper C++ pour FreeRTOS
  \item Développement interface graphique: Modèle Vue-Controller
  \item Intégration stack logiciel Segger: emWin, emFile, emUSB
\end{itemize}}

% Janv 2015 - Sept 2015
\cventry{2015--2015\newline9 mois}{Responsable logiciel embarqué du projet Linky G1 Triphasé}{Itron}{Poitiers(86)}
	{Pilotage développement logiciel et management d’équipe, 3 ingénieurs. Report d’activité et avancement au chef de projet}
	{Conception et développement. Mise en place de test d’intégration. Support technique}
\subsection{Contexte}
\cvline{}{
\begin{itemize}
  \item 
\end{itemize}}
\subsection{Activités}
\cvline{}{
\begin{itemize}
  \item 
\end{itemize}}

% Oct 2011 - Dec 2014
\cventry{2011--2014\newline3 ans}{Ingénieur logiciel embarqué temps réel}{Itron}{Poitiers(86)}
	{Conception et développement de fonctionnalités pour compteur électrique Linky}
	{C++03, STM32, RTOS RL-RTX, Eclipse, Keil uVision, Perforce}
\subsection{Contexte}
\cvline{}{
\begin{itemize}
  \item Projet compteur électrique communicant Linky pour EDF
\end{itemize}}
\subsection{Activités}
\cvline{}{
\begin{itemize}
  \item Composant Metering: Sauvegarde consommation par type de grandeur physique, Papp, Pactive, Preactive, ...
  \item Composant Contacteur: Dev driver, pilotage des relais du contacteur, Organe de sécurité (surpuissance, surchauffe), 
  anti-fraude (magnétique, injection de courant), commande à distance, commande utilisateur (bouton poussoir)
  \item Composant Demand: ...
  \item Test Intégration: Outils de script Itron (Manitoo)
  \item RAM Monitor: Analyse de consommation RAM par tache RTOS
  \item Support technique
\end{itemize}}

% Sept 2010 - Sept 2011
\cventry{2010--2011}{Apprenti par alternance}{ST-Ericsson}{Sophia-Antipolis (06)}
	{Conception et développement d’un système de diagnostique matériel pour ARM9 embarqué dans les téléphones}
	{C99/C++03, Trace32, Lauterbach, Eclipse, QT 4}
%\closesection{}

\section{Extra}
\subsection{Objectif}
\cvline{}{
\begin{itemize}
	\item Recherche opportunité d'évolution et obtenir plus de responsabilité
	\item Être acteur dans la prise de décision, moteur dans l'entreprise
	\item S'investir dans une entreprise pour du long terme
	\item Quitter la prestation pour un client final
\end{itemize}}
\cvline{Design Pattern}{Singleton, state machine, observer, Interface, MVC}

% Footnote at the end to add version
\emptysection{}\closesection
\vfill
\begin{center}
\textit{\small 2021.10}
\end{center}

\end{document}
